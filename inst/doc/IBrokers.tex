%\VignetteIndexEntry{IBrokers: Interactive Brokers and R}
\documentclass{article}
\usepackage{hyperref}
\hypersetup{colorlinks,%
            citecolor=black,%
            linkcolor=blue,%
            urlcolor=blue,%
            }

\title{\bf IBrokers - Interactive Brokers and R }
\author{Jeffrey A. Ryan}
\date{May 6, 2008}

\usepackage{/Users/jryan/Desktop/R-2.7.0/share/texmf/Sweave}
\begin{document}

\maketitle
\tableofcontents

\begin{abstract}
The statistical language {\tt R}
offers a great environment for rapid trade
idea development and testing. Interactive Broker's
\emph{Trader Workstation} offers a
robust platform for execution of these ideas.
Previously it was required to use an external
language to interface the impressive API
capabilities of the \emph{Trader Workstation} --- be
it Java, Python, C++, or a myriad of other
language interfaces, both officially supported
or otherwise.
What had been lacking was a native {\tt R} interface
to access this impressive API.  This is now available
in the new IBrokers package.
\end{abstract}

\section{Introduction}
\textcolor{red}{
This software is in no way affiliated, endorsed, or approved by
     Interactive Brokers or any of its affiliates. It comes with
     absolutely no warranty and should not be used in actual trading
     unless the user can read and understand the source.
}

Interactive Brokers \cite{TWS} is an international brokerage
firm specializing in electronic execution in
products ranging from equities to bonds, options to futures,
as well as FX, all from a single account. To facilitate
this they offer access to their \emph{Trader Workstation}
platform (TWS) through a variety of proprietary APIs.
The workstation application is written
in Java and all communication is handled via sockets
between the client and the TWS.

Access to the API has official support in
Java (all platforms), as well as C++, DDE for Excel, and ActiveX
(available only for Windows). There are numerous third-party
applications and API interfaces available to access the
API. Some include IbPy - a python translation of the
official Java code, a Perl version, and a portable C version.

All of these methods, while useful outside of {\tt R}~\cite{R}
can't offer true R-level access to the TWS API.  For this an
R solution was required.

This introduction is broken into two parts. First
it will provide an overview of the overall
API from Interactive Brokers, as well as examine some of the
more common documented methods. The second section will
examine the specific implementation of this API in the
IBrokers \cite{IBrokers} package.

\section{The API}
The most up to date documentation on the overall API can be found
on Interactive Brokers own site. While it is a constantly evolving
library - most of the core functionality persists from
one version to the next. The principal purpose of the API is to
offer a programmatic alternative to manual screen-based trading
through Interactive Brokers.

\subsection{Data}
In order for trade decisions to be automated, data
must be processed by the client application. To retrieve
real-time data from the TWS there are three primary
access methods - {\tt reqMktData},
{\tt reqMktDepth}, and {\tt reqRealTimeBars}.  Additionally,
limited historical data can be retrieved for many products
via the {\tt reqHistoricalData} accessor function. All of these
operate with callback handlers within the public API - so
it is possible to allow for custom actions based on the
incoming data.

New in the beta version of the API are tools to access
fundamental data from Thompson Financial.  These are
not in production versions as of this writing.

\subsection{Execution}
The API also allows for order execution to be programmatically
driven. Through a variety of functions, it is possible
to view, modify, and submit orders to be executed by
the TWS.

\subsection{Miscellaneous functionality}
Additional functionality offered includes access
to account information, contract details, connection
status and news bulletins from the TWS.

\section{IB and R}
As R offers an ever-growing tooklit of statistical as well
as financial functionality, it is becoming a platform
of choice for quantitative research and even trading. For
institutional clients many tools exists to help
tie data from external sources into R. Probably the most
common is the use of Bloomberg data in R via the
RBloomberg package.  While many professionals have
access to a Bloomberg terminal, it is not really practical
or necessary for smaller or single product traders.

Interactive Brokers gives these users access to many
professional features - as well as industry leading
executions and prices - all from a solid GUI
based application.

To make the transition to programmatically interacting
with this application, it had been necessary to use
one of the supported or contributed API libraries
available. For many users this poses no issue --- as many
are either unaware of R as a quantitative platform, or
make use of it in a limited manner.

For those that use R more frequently it is important
to find a workable solution accessible from within
R.  This is the purpose of IBrokers.

\subsection{Getting started}
The first step in interacting with the TWS is to
create a connection.  To do so, it is first
necessary to enable the TWS to allow for
incoming socket connections. The TWS
user manual is the most up to date reference on
how to accomplish this, but as of this document
it is simply a matter of \textbf{Configure > API > Enable ActiveX and Sockets}.
You should also add your machine (127.0.0.1) to the \textbf{Trusted IP Addresses}.

From the R prompt, load the package and establish a
connection. Once connected you can then check some basic connection information.
Multiple TWS connections are allowed per client, though a distinct
clientId is required, and given R's single threaded nature it is not
of much value.
If no further queries are required, {\tt twsDisconnect} will disconnect
the R session from the TWS.
\begin{Schunk}
\begin{Sinput}
> library(IBrokers)
> tws <- twsConnect()
> tws
> reqCurrentTime(tws)
> serverVersion(tws)
> twsDisconnect(tws)
\end{Sinput}
\end{Schunk}

\subsection{Getting data from the TWS}
The current IBrokers implementation is focused on
retrieving data from the TWS. To that end, four
functions are made available in the API:
\begin{quote}
\begin{description}
\item[reqMktData:] retrieves real-time market data.
\item[reqMktDepth:] retrieves real-time order book data.
\item[reqRealTimeBars:] retrieves real-time OHLC data.
\item[reqHistoricalData:] retrieves historical data.
\end{description}
\end{quote}
In addition, due to the complexity of requests
helper functions unique to the {\tt R} API
are included to make constructing these faster
and less error-prone. A family of contract
specifier functions used for the above calls
includes:
\begin{quote}
\begin{description}
\item[twsContract:] create a general Contract object.
\item[twsEquity:] wrapper to create equity Contract objects
\item[twsOption:] wrapper to create option Contract objects.
\item[twsFuture:] wrapper to create futures Contract objects.
\end{description}
\end{quote}
Unique again to the IBrokers implementation is the
use of passed callback handlers to the data functions. These are
available to help customize output and in the creation of
automated trading programs based on user-defined criteria
coded in R.

Each function has a file argument which allows for persistent
data capture to a file.  In the case of the real-time methods
file is passed internally to {\tt cat}, which mean that the
argument may contain a file name or any valid connection
object.

Each data function has special callback methods
associated with the expected results. These are documented
in the standard R help files of IBrokers.
Each distinct message recieved from the TWS
invokes a callback function - by default the callbacks
built into IBrokers. Users may set these callbacks to
functions of there own design, or set them to NULL to
return the raw message data.

Additionally, the
real-time data methods {\tt reqMktData}, {\tt reqMktDepth}, and
{\tt reqRealTimeBars} all have a special CALLBACK
argument to allow for custom raw message handling. This allows for
simplified management of data calls, while offering highly
customizable receiver handling. Using this will
bypass the built in error management of TWS errors, so the
developer should be aware of low-level API interactions
before using this feature.

\subsection{Future API access}
Future additions will focus on implementing a robust order
management collection of functions to allow for a complete
data--processing--trade workflow.

\section{Conclusion}
The TWS platform offers the quantitative trader unparalleled
access to automated trading tools via its API.  IBrokers
allows the {\tt R} developer direct access to this API
for rapid trade idea development, testing, and eventual execution.

\begin{thebibliography}{99}
\bibitem{TWS} Interactive Brokers:
\emph{Interactive Broker's \emph{Trader Workstation} and API},
URL http://www.interactivebrokers.com

\bibitem{R} R Development Core Team:
\emph{R: A Language and Environment for Statistical Computing},
R Foundation for Statistical Computing, Vienna, Austria.
ISBN 3-900051-07-0, URL http://www.R-project.org

\bibitem{IBrokers} Jeffrey A. Ryan:
\emph{IBrokers: R API to Interactive Brokers Trader Workstation},
R package version 0.1-0, 2008
\end{thebibliography}
\end{document}
